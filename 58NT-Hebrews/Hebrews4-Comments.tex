\section{Hebrews 4 Comments}

\subsection{Numeric Nuggets}
\textbf{13:} The 13$^{th}$ word in the chapter starts the phrase ``into rest.''

\subsection{Hebrews 4 Introduction}

Hebrews 4 is a straightforward passage, made difficult by modern Bible expositors who provide shallow exegesis! For light, the referenced passage is Psalm 95, quoted in verse 7 and referenced  in verse 3. Events referenced are the creation in verse 4, the rebellion on Israel at Kadesh-Barnea in Numbers 13:26-33, and something called a ``rest''. There is also a ``gospel'' that shows up in verse 2.\footnote{\textbf{Numbers 13:26-33} - And they went and came to Moses, and to Aaron, and to all the congregation of the children of Israel, unto the wilderness of Paran, to Kadesh; and brought back word unto them, and unto all the congregation, and shewed them the fruit of the land. [27] And they told him, and said, We came unto the land whither thou sentest us, and surely it floweth with milk and honey; and this is the fruit of it. [28] Nevertheless the people be strong that dwell in the land, and the cities are walled, and very great: and moreover we saw the children of Anak there. [29] The Amalekites dwell in the land of the south: and the Hittites, and the Jebusites, and the Amorites, dwell in the mountains: and the Canaanites dwell by the sea, and by the coast of Jordan. [30] And Caleb stilled the people before Moses, and said, Let us go up at once, and possess it; for we are well able to overcome it. [31] But the men that went up with him said, We be not able to go up against the people; for they are stronger than we. [32] And they brought up an evil report of the land which they had searched unto the children of Israel, saying, The land, through which we have gone to search it, is a land that eateth up the inhabitants thereof; and all the people that we saw in it are men of a great stature. [33] And there we saw the giants, the sons of Anak, which come of the giants: and we were in our own sight as grasshoppers, and so we were in their sight.}

An important distinction to be made, first of all, is between four distinct ``rests'' discussed in Hebrews 3 and 4:
\begin{compactenum}
	\item There is the ``Canaan Rest'' in Hebrews 3:16-19.  Canaan, aka the Promised Land, was available to Israel in Numbers 13, but because of unbelief, they refused to obey God and enter.
	\item There is ``Creation Rest'' in Hebrews 4:3-4, when God rested on the seventh day. Was God tired? Did he need a break? No. This ``sabbath'' rest was a sign to Israel (see Exodus 20:20), and pointed to the Millennium.\footnote{\textbf{Ezekiel 20:20} - And hallow my sabbaths; and they shall be a sign between me and you, that ye may know that I am the LORD your God.}
	\item There is the ``Millennial Rest'' in Hebrews 4:7-8, the 1,000-year reign of Christ on Earth.
	\item There is the ``Believer's Rest'' in Hebrews 4:9-10. 
\end{compactenum}
In Numbers 13 and 14 the Israelites came short of a rest, a the offer was not received with faith,but unbelief.


\subsection{Hebrews 4:2}
Having been accused at times of ``reading things into scripture'', I notice the phenomenon where any time the word ``gospel'' is used in scripture, brethren ``read into'' the word the gospel as found in 1 Corinthians 15:1-4, or better yet, ``read out'' any other possible interpretation.  The word ``gospel'' simply means  ``good news'', and I seriously doubt that Moses was preaching 1 Corinthians 15:1-4 in Numbers 13.  Paul's gospel could not apply in Numbers 13 because it simply was not yet available. What specifically, then, was preached to the Israelites in the wilderness, then? Which gospel is being spoken of here?

As pointed out by a few, the New Testament contains no less than ten separate gospels. The one spoken of here in Hebrews 4:2, is \#8 in the list below:
\begin{compactenum}
	\item The gospel of Matthew
	\item The gospel of Mark
	\item The gospel of Luke
	\item The gospel of John
	\item The gospel of the kingdom
	\item The gospel of the grace of God
	\item The glorious gospel of the blessed God	(also called ``my gospel'' by Paul).\footnote{\textbf{Romans 2:16} - In the day when God shall judge the secrets of men by Jesus Christ according to my gospel.}\footnote{\textbf{1 Timothy 1:11} - According to the glorious gospel of the blessed God, which was committed to my trust.}
	\item The gospel of military conflict preached to the nation of Israel: God would give them victory in the military conquest of the land of Canaan to the extent that they obeyed Him (Numbers 13:30, 14:6-10, Hebrews 4:2). \footnote{\textbf{Numbers 14:6-10} - And Joshua the son of Nun, and Caleb the son of Jephunneh, which were of them that searched the land, rent their clothes: [7] And they spake unto all the company of the children of Israel, saying, The land, which we passed through to search it, is an exceeding good land. [8] If the LORD delight in us, then he will bring us into this land, and give it us; a land which floweth with milk and honey. [9] Only rebel not ye against the LORD, neither fear ye the people of the land; for they are bread for us: their defence is departed from them, and the LORD is with us: fear them not. [10] But all the congregation bade stone them with stones. And the glory of the LORD appeared in the tabernacle of the congregation before all the children of Israel.}
	\item The gospel preached to the saints in paradise.\footnote{\textbf{1 Peter 4:6} - For for this cause was the gospel preached also to them that are dead, that they might be judged according to men in the flesh, but live according to God in the spirit.}
	\item The everlasting gospel.\footnote{\textbf{Revelation 14:6} - And I saw another angel fly in the midst of heaven, having the everlasting gospel to preach unto them that dwell on the earth, and to every nation, and kindred, and tongue, and people,}
\end{compactenum}

\subsection{Hebrews 4:4}
See 2 Peter 3:8-10 and the 7$^{th}$ day of creation.\footnote{\textbf{2 Peter 3:8-10} - But, beloved, be not ignorant of this one thing, that one day is with the Lord as a thousand years, and a thousand years as one day. [9] The Lord is not slack concerning his promise, as some men count slackness; but is longsuffering to us-ward, not willing that any should perish, but that all should come to repentance. [10] But the day of the Lord will come as a thief in the night; in the which the heavens shall pass away with a great noise, and the elements shall melt with fervent heat, the earth also and the works that are therein shall be burned up.} 

\subsection{Hebrews 4:8}
Jesus here is speaking of Joshua.

\subsection{Hebrews 4:12}
stuff.%\footnote{}

\subsection{Hebrews 4:15}
This is empathy.
